\documentclass{beamer}
\usepackage[T1]{fontenc}
\usepackage[utf8]{inputenc}
\usepackage[italian]{babel}
\usepackage{lmodern}
\usepackage{amsmath}
\usepackage{amsfonts}
\usepackage{amssymb}
\usepackage{amsthm}
\usepackage{graphicx}
\usepackage{color}
\usepackage{xcolor}
\usepackage{url}
\usepackage{theorem}
\usepackage{textcomp}
\usepackage{listings}
\usepackage{hyperref}
\usepackage[thicklines]{cancel}
\usepackage{parskip}     
\usetheme{Madrid}



\title{What The Helm?!}
\subtitle{Panoramica su Helm}
\author{Luca Andrea Fusè}
\date{21 Ottobre 2023}

\begin{document}

\renewcommand{\CancelColor}{\color{red}}

\begin{frame}
    \titlepage
\end{frame}
 
 \begin{frame}
     \frametitle{Presentazione}

    \begin{itemize}
        \item Membro (onorario) dell'associazione \href{https://www.poul.org}{P.O.u.L.}
        \item Socio di \href{https://www.ils.org}{I.L.S.} nella sezione di Milano
        \item Consulente per \href{https://www.kyndryl.com}{Kyndryl}
        \begin{itemize}\item TL;DR Faccio cose con Openshift/Kubernetes/Cloud \end{itemize}
        \item Sono \cancel{1}0 giorni che non parlo di Kubernetes e/o Linux
    \end{itemize}
 \end{frame}
 
 \begin{frame}
   \frametitle{Di cosa parleremo}
   \begin{enumerate}
    \item Cosa fa e quali problemi risolve?
    \item Anatomia di un Helm Chart
    \item Quando Helm ci fa dire "WTH?!"
   
   \end{enumerate}
  \end{frame}

\begin{frame}
\frametitle{Cosa è Helm}
 \begin{itemize}
    \item Helm è un tool che permette di configurare e deployare applicazioni su Kubernetes
    \item Semplifica il deploy sfuttando l'uso di \textbf{variabili} e \textbf{template}
    \item In poche parole, Helm è \textit{il gestore di pacchetti} per Kubernetes
  \end{itemize}
\end{frame}

 \begin{frame}
     
 \end{frame}

\end{document}
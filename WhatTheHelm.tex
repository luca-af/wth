\documentclass{beamer}
\usepackage[T1]{fontenc}
\usepackage[italian]{babel}
\usepackage{lmodern}
\usepackage{amsmath}
\usepackage{amsfonts}
\usepackage{amssymb}
\usepackage{amsthm}
\usepackage{graphicx}
\usepackage{color}
\usepackage{xcolor}
\usepackage{url}
\usepackage{theorem}
\usepackage{textcomp}
\usepackage{listings}
\usepackage{hyperref}
\usepackage[thicklines]{cancel}
\usepackage{parskip}     
\usepackage{listings}
\usetheme{Madrid}
\usepackage{courier}

%% Customization
 \hypersetup{
     colorlinks=true,
     linkcolor= black,
     filecolor=blue,
     citecolor = black,      
     urlcolor=blue,
     }
\lstset{basicstyle=\footnotesize\ttfamily,breaklines=true}
\lstset{framextopmargin=50pt,frame=bottomline}
\setbeamertemplate{frametitle continuation}{}
\lstdefinelanguage{yaml}{
  keywords={spec, containers, name, description, dependencies, repository, version, appVersion, apiVersion, tag, pullPolicy, pullSecrets, debug, image, livenessProbe, enabled, httpGet, metadata, kind, imagePullPolicy}, % ,... all the keyword you want
  keywordstyle=\color{blue}\bfseries,
  moredelim=[is][commentstyle]{||}{££}, % invisible custom delimiters
  identifierstyle=\color{black},
  sensitive=false,
  comment=[l]{\#},
  commentstyle=\color{olive}\ttfamily,
  stringstyle=\color{orange}\ttfamily,
  morestring=[b]',
  morestring=[b]"
}

\setbeamertemplate{section page}
{
    \begingroup
    \begin{beamercolorbox}[sep=12pt,center]{section title}
        \usebeamerfont{section title}\insertsection\par
    \end{beamercolorbox}
    \endgroup
}

\AtBeginSection[]{
    \begin{frame}[plain]
        \usebeamertemplate{section page}
    \end{frame}
}

%% Document's info

\title{What The Helm?!}
\subtitle{Panoramica su Helm}
\author{Luca Andrea Fusè}
\date{28 Ottobre 2023}

\begin{document}

\renewcommand{\CancelColor}{\color{red}}

\begin{frame}
    \titlepage
\end{frame}
 
 \begin{frame}
     \frametitle{Presentazione}

    \begin{itemize}
        \item Membro onorario dell'associazione \href{https://www.poul.org}{P.O.u.L.}
        \item Socio di \href{https://www.ils.org}{I.L.S.} nella sezione di Milano
        \item Consulente per \href{https://www.kyndryl.com}{Kyndryl}
        \begin{itemize}\item TL;DR Faccio cose con OpenShift/Kubernetes/Cloud \end{itemize}
        \item Sono \cancel{1}0 giorni che non parlo di Kubernetes e Linux
    \end{itemize}
 \end{frame}
 
 \begin{frame}
   \frametitle{Di cosa parleremo?}
   \tableofcontents
%   \begin{enumerate}
%    \item Cosa fa e quali problemi risolve?
%    \item I comandi per iniziare
%    \item Anatomia di un Helm Chart (versione \textbf{molto} concentrata)
%    \item Quando Helm ci fa dire "WTH?!"
   
%   \end{enumerate}
  \end{frame}
  
\section{Cosa fa e quali problemi risolve?}
\begin{frame}
\frametitle{Cosa fa?}
 \begin{itemize}
    \item Helm è un tool che permette di configurare e deployare applicazioni su Kubernetes
    \item Semplifica il deploy sfuttando l'uso di \textbf{variabili} e \textbf{template}
    \item In poche parole, Helm è "il gestore di pacchetti" per Kubernetes
  \end{itemize}
\end{frame}

 \begin{frame}
 \frametitle{Quali problemi risolve?}
 \begin{itemize}
     \item Voglio avere la certezza che i deploy che faccio siano \textit{sempre uguali}
     \item Voglio avere la possibilità di riutilizzare e condividere il \cancel{template} Helm Chart che ho creato
     \item Voglio fare un deploy di più applicazioni con dipendenze tra di loro
     \item Voglio che sia anche possibile gestire un update/rollback di quanto deployato
 \end{itemize}
 \end{frame}
 
\section{Un minimo di terminologia...} 
 \begin{frame}[allowframebreaks]{Un minimo di terminologia...}
 
 \begin{block}{Helm}
 Tool da cli che ci permette di creare e gestire gli \textbf{Helm Chart}
 \end{block}
 \framebreak
 \begin{block}{Helm Chart}
 Progetto che contiene informazioni, template di risorse ed eventuali valori di default che vengono processati quando viene fatto un deploy.
 \end{block}
 \framebreak
 \begin{block}{Template di una risorsa}
 Template (Go template) che, in base alle logiche con cui è stato scritto e alle variabili passata, va ad essere convertito nella definizione della risorsa
 \end{block}
 \framebreak
 \begin{block}{Helm repository}
 Sorgente dalla quale vengono scaricati gli Helm Chart
 \end{block}
 \end{frame}
 
 \section{Comandi per iniziare}
 \begin{frame}[allowframebreaks,fragile]
 \frametitle{Comandi per iniziare}
 \begin{block}{Helm repo}
  Con il comando \lstinline[language=bash]{helm repo add} possiamo aggiungere un repository da cui pescare gli Helm Chart
  \end{block}
  \framebreak
  \begin{block}{Esempio}
  \begin{lstlisting}[backgroundcolor=\color{white},language=bash,basicstyle=\small]
  $ helm repo add  \
    bitnami https://charts.bitnami.com/bitnami
  \end{lstlisting}
  \end{block}
  \framebreak
  
  \begin{block}{Helm search repo}
  Il comando \lstinline[language=bash]{helm search repo} ci permette di cercare nei repository aggiunti un determinato termine
  \end{block}
  
  \framebreak
  \begin{block}{Esempio}
  \begin{lstlisting}[backgroundcolor=\color{white},language=bash,basicstyle=\small]
  $ helm search repo mysql
    NAME                  CHART VERSION APP VERSION                                       
    bitnami/mysql         9.14.1        8.0.35     
    bitnami/phpmyadmin    13.0.0        5.2.1      
    bitnami/mariadb       14.1.0        11.1.2     
  \end{lstlisting}
  \end{block}
  \framebreak
  
  \begin{block}{Helm Install}
  Il comando \lstinline[language=bash]{helm install} permette di deployare l'applicazione/le applicazioni definite nel nostro chart
  \end{block}
  
  \framebreak
  \begin{block}{Esempio}
  \begin{lstlisting}[backgroundcolor=\color{white},language=bash,basicstyle=\small]
  $ helm install \
    -f myvalues.yaml --set mariadb.auth.database=userdb \ 
    my_wordpress bitnami/wordpress
  \end{lstlisting}
  \end{block}
 \end{frame}
\section{Anatomia di un Helm Chart (\textbf{molto} concentrata)}
 \begin{frame}[allowframebreaks,fragile]{Anatomia di un Helm Chart}
  Abbiamo già accennato a che cosa sia un Helm Chart, ma: 
  \begin{itemize}
      \item Come faccio a crearne uno?
      \item Da cosa è composto?
      \item Come sono fatti i template al loro interno?
  \end{itemize}
  \framebreak
  \begin{block}{Creazione di un Helm Chart}
  Per creare un Helm Chart posso facendolo a mano oppure con il comando \lstinline[language=bash]{helm create} 
  \end{block}
  \framebreak
  \begin{block}{Esempio}
  \begin{lstlisting}[backgroundcolor=\color{white},language=bash,basicstyle=\small]
  $ helm create mychart
  $ ls mychart
  .helmignore
  Chart.yaml
  values.yaml
  charts/
  templates/
  \end{lstlisting}
  \end{block}
  \framebreak
  \begin{block}{Da cosa è composto?}
  Un helm chart è composto da diverse file, di cui i principali sono:
  \begin{itemize}
      \item \lstinline[language=bash]{Chart.yaml}
      \item \lstinline[language=bash]{values.yaml}
      \item \lstinline[language=bash]{templates/}
  \end{itemize}
  \end{block}
  \framebreak
  \begin{block}{Chart.yaml}
  \begin{lstlisting}[backgroundcolor=\color{white},language=yaml,basicstyle=\tiny]
apiVersion: v2
name: wordpress
description: WordPress is the world...
[...]
dependencies:
  - name: mariadb
    repository: [...]
    version: 14.x.x
[...]
version: 18.0.10
appVersion: 6.3.2
  \end{lstlisting}
  \end{block}
  \framebreak
  \begin{block}{values.yaml}
  \begin{lstlisting}[backgroundcolor=\color{white},language=yaml,basicstyle=\tiny]
  
[...]
image:
  registry: docker.io
  repository: bitnami/wordpress
  tag: 6.3.2-debian-11-r3
  pullPolicy: IfNotPresent
  pullSecrets: []
  debug: false
[...]
livenessProbe:
  enabled: true
  httpGet:
    path: /wp-admin/install.php
    port: '{{ .Values.wordpressScheme }}'
    scheme: '{{ .Values.wordpressScheme | upper }}'
    httpHeaders: []
[...]
  \end{lstlisting}
  \end{block}
 \framebreak
 \begin{block}{template/deployment.yaml}
 \begin{lstlisting}[backgroundcolor=\color{white},language=yaml,basicstyle=\tiny]
apiVersion: {{ include "common.capabilities.deployment.apiVersion" . }}
kind: Deployment
metadata:
  name: {{ include "common.names.fullname" . }}
  namespace: {{ .Release.Namespace | quote }}
[...]
  containers:
    - name: wordpress
      image: {{ include "wordpress.image" . }}
      imagePullPolicy: {{ .Values.image.pullPolicy | quote }}
      {{- if .Values.diagnosticMode.enabled }}
      command: >
        {{- include "common.tplvalues.render" 
        (dict "value" .Values.diagnosticMode.command "context" $) | nindent 12 }}
      {{- else if .Values.command }}
      command: >
        {{- include "common.tplvalues.render" 
        ( dict "value" .Values.command "context" $) | nindent 12 }}
      {{- end }}
[...]
\end{lstlisting}
\end{block}
\end{frame} 
\section{Quando Helm ci fa dire WTH?!}
\begin{frame}[allowframebreaks,fragile]{WTH?!}
\begin{block}{Cosa può andare storto?}
Per quanto Helm sia una ottima soluzione, a volte può farci tirare alcuni \cancel{WTF?!} WTH?!.
\end{block}
\framebreak
\begin{block}{Template di Helm}
 Il motore che è stato scelto per i template è quello di Go. \newline
 I suoi costrutti, assieme alla sintassi yaml delle risorse k8s, rendono la scrittura e la lettura dei template di Helm non facilissima.  
\end{block}
\framebreak
\begin{block}{Dipendenze}
Con la sezione \lstinline{dependencies} di Chart.yaml posso andare a dichiarare gli Helm Chart da cui il nostro dipende. \newline 
Questo porta ovviamente a possibili bug/incompatibilità in base ai Chart che sto utilizzando e su cui non ho il controllo.
\end{block}
\framebreak
\begin{block}{Complessità}
Anche se abbiamo un martello, non vuol dire che tutto debba essere un chiodo. \newline Per quanto Helm sia un ottimo strumento, non è detto che per il nostro deploy sia uno strumento sensato.
\end{block}
\framebreak 
\begin{block}{Complessità 2}
Poter far in modo che il nostro Helm Chart dichiari gli altri Chart di cui necessita è \textbf{molto} comodo. \newline Però bisogna far attenzione a non lasciarsi prendere (troppo) la mano!
\end{block}
\framebreak
\begin{block}{Sicurezza}
Quando si utilizzano Helm Chart provenienti dalla community, bisogna fare attenzione ad utilizzare Chart che si reputano affidabili. \newline
Questo si risolve facilmente utilizzando Chart ufficiali oppure scrivendone di nostri custom.
\end{block}
\end{frame}
\begin{frame}{}
\centering \Large \emph{Domande?}
\end{frame}
\begin{frame}{Links utili}
\begin{itemize}
    \item \href{https://helm.sh/docs/}{Documentazione di Helm}
    \item \href{https://github.com/bitnami/charts/tree/main/bitnami/wordpress}{Helm Chart} da cui sono stati tratti gli snippet
    \item \href{https://artifacthub.io/}{Artifactor Hub} da cui si possono recuperare diversi Helm Chart (oltre a diversi altri tipi di oggetti)
\end{itemize}
\end{frame}
\begin{frame}{}
\centering \Large \textbf{Grazie per l'attenzione!}
\end{frame}

\begin{frame}{Licenza slides}
\centering
Queste slides sono rilasciate sotto \href{https://creativecommons.org/licenses/by-sa/4.0/}{CC-BY-SA 4.0}
\begin{center} \includegraphics{img/by-sa.png} \end{center}
\end{frame}
\end{document}
